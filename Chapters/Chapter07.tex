\myChapter{Podsumowanie}\label{ch:conclusion}
%************************************************

Udało mi się zrealizować założenia, które przewidywała moja praca inżynierska.

Niestety nie udało się zawrzeć dodatkowych funkcjonalności, które znacznie zwiększyłyby możliwości systemu i sprawiłyby, że stałby się znacznie bardziej ciekawą propozycją.

Chociaż konstrukcja sprzętu nie należy do zakresu tej pracy, to możliwość udziału w projektowaniu całego systemu stanowiła przyjemną odskocznię od programów tworzonych na zajęcia na uczelni. Praca z systemem dała dużą satysfakcję ze względu na działanie z realnym urządzeniem, a nie tylko i wyłącznie algorytmem zapisanym w pamięci komputera.

%Uważam, że moja praca jest interesująca również dlatego, że działa ona po części także poza komputerem, co powoduje,

Podsumowując pracę związaną z projektem, zamieszczam listę wad i zalet systemu.

Główne wady to:
\begin{aenumerate}
 \item niestabilność \ppauza próbki wykazują relatywnie duże zmiany pomimo stabilności markerów,
 \item zawodność \ppauza aby system działał sprawnie, wymagane jest celowanie markerem w stronę odbiorników, co uniemożliwia sprawne wykorzystywanie systemu bez obawy o zgubienie pozycji,
 \item mała szybkość \ppauza firmy produkujące myszki i klawiatury dla graczy prześcigują się w technologiach możliwie najszybszych czasów reakcji, natomiast \textsl{Nietoperz} osiąga szybkość zaledwie 25Hz; nawet jeśli byłaby możliwość przyspieszenia, to i tak pozostawałaby ta wartość daleko w tyle za 100Hz, którą mogą poszczycić się prawdziwe kontrolery,
 \item możliwość korzystania tylko przez jedną osobę \ppauza eliminuje możliwość gry zespołowej,
 \item konieczność kalibracji \ppauza może zniechęcać do korzystania z systemu, ponadto przypadkowe przesunięcia \textsl{Nietoperza} względem użytkownika mogą być odczytywane jako niepożądany ruch.
\end{aenumerate}

Do zalet systemu można natomiast zaliczyć:
\begin{aenumerate}
 \item niskie koszty \ppauza odtworzenie całości systemu jest na każdą kieszeń,
 \item innowacyjność \ppauza wykorzystałem metodę, jaka nie była zastosowana do tej pory w konsolach do gier,
 \item prostota \ppauza zasada działania \textsl{Nietoperza} opiera się o podstawową geometrię, łatwo więc zrozumieć metodę jego działania.
\end{aenumerate}

%Przeprowadzone porównanie, a także korzystanie z systemu, pokazuje, że nie jest on wystarczająco wydajny, a także dostarcza niedostatecznie stabilnych wyników, aby można było go stosować jako system manipulacji.

Z przykrością muszę stwierdzić, że wady systemu przeważają jego zalety, zatem uznaję go za niezdatny jako interfejs człowiek-komputer, a w szczególności jako kontroler do gier. \textsl{Nietoperz} pozostanie jednak interesującym eksperymentem, który być może znajdzie jeszcze jakieś niszowe zastosowanie.