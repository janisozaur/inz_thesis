%************************************************
\myChapter{Aktualny stan zagadnienia}\label{ch:current_state} % $\mathbb{ZNR}$
%************************************************

Od kilku lat można zaobserwować wprowadzanie na rynek masowy, głównie jako elementów konsol do gier, systemów śledzenia ruchu.

Zadaniem takiego systemu jest śledzenie użytkownika w pewnej ograniczonej przestrzeni, w szczególności jego ruchów, oraz przetwarzanie pozyskanych danych na potrzeby gry wideo. W efekcie gracz wykazuje większe zaangażowanie w grę, ponieważ staje się jej aktywną częścią.

Pierwsze próby śledzenia ruchów podjęła firma \textsmaller{abrams gentile entertainment} w roku 1989 wprowadzając rękawicę \textsmaller{Power Glove}, która miała na celu śledzenie ruchów ręki użytkownika i wykorzystanie tych danych w konsoli \index{Nintendo!NES}\textsmaller{Nintendo Entertainment System}.

Z powodu słabego przyjęcia się urządzenia (miała na to wpływ między innymi jakość urządzenia oraz ilość dostępnych gier wykorzystujących możliwości tego kontrolera) kolejne próby śledzenia ruchów podjęła firma \index{Sony}\textsmaller{Sony} wprowadzając kamerę \textsmaller{EyeToy} dopiero w roku 2003. Ów odstęp czasu był długi również z powodu wymagań technologicznych, jakim musiał sprostać sprzęt aby nie tylko generować, ale także pobierać i przetwarzać obraz w czasie rzeczywistym. Wspomniane urządzenie także nie cieszyło się dużą popularnością.

Pomimo chłodnego przyjęcia dotychczasowych rozwiązań, producenci dostrzegali duże możliwości rozwoju w tej dziedzinie, gdyż w roku 2006. wprowadzona została przez firmę \index{Nintendo}\textsmaller{Nintendo} konsola \index{Nintendo!Wii}\textsmaller{Wii}, której głównym atutem był nowy sposób sterowania \ppauza za pomocą bezprzewodowego kontrolera wyposażonego w akcelerometr. Urządzenie to jest w stanie odczytywać przyspieszenia, jakie na nie działają we wszystkich trzech osiach. Wprowadzony w roku 2009. dodatek \textsmaller{MotionPlus}, który zawiera trójosiowy żyroskop, umożliwia dodatkowo śledzenie obrotów w trzech osiach, znacznie zwiększając dokładność systemu.

Na dzień dzisiejszy, wśród aktualnych ,,dużych''\graffito{Mowa tutaj o systemach, które nie są projektowane jako przenośne, chociaż i w tym przypadku zaczyna się wykorzystywać jakieś formy śledzenia ruchów.} konsol siódmej generacji, każdy z domowych systemów posiada system śledzenia ruchu. Są to, w kolejności wprowadzania na rynek:
\begin{enumerate}
 \item \index{Nintendo!Wiimote}\textsmaller{Wiimote} \ppauza \textsmaller{Nintendo Wii},
 \item \index{Sony!PlayStation!Eye}\textsmaller{PlayStation Eye} \ppauza \textsmaller{Sony PlayStation},
 \item \index{Nintendo!MotionPlus}\textsmaller{Wii MotionPlus} \ppauza \textsmaller{Nintendo Wii},
 \item \index{Sony!PlayStation!Move}\textsmaller{PlayStation Move} \ppauza \textsmaller{Sony PlayStation},
 \item \index{Microsoft!Kinect}\textsmaller{Kinect} \ppauza \textsmaller{Microsoft Xbox 360}.
\end{enumerate}

% Ei choro aeterno antiopam mea, labitur bonorum pri no. His no decore
% nemore graecis. In eos meis nominavi, liber soluta vim cu. Sea commune
% suavitate interpretaris eu, vix eu libris efficiantur.
% 
% \section{Some Formulas}
% Due to the statistical nature of ionisation energy loss, large
% fluctuations can occur in the amount of energy deposited by a particle
% traversing an absorber element\footnote{Examples taken from Walter
% Schmidt's great gallery: \\
% \url{http://home.vrweb.de/~was/mathfonts.html}}.  Continuous processes
% such as multiple
% scattering and energy loss play a relevant role in the longitudinal
% and lateral development of electromagnetic and hadronic
% showers, and in the case of sampling calorimeters the
% measured resolution can be significantly affected by such fluctuations
% in their active layers.  The description of ionisation fluctuations is
% characterised by the significance parameter $\kappa$, which is
% proportional to the ratio of mean energy loss to the maximum allowed
% energy transfer in a single collision with an atomic electron:
% \graffito{You might get unexpected results using math in chapter or
% section heads. Consider the \texttt{pdfspacing} option.}
% \[
% \kappa =\frac{\xi}{E_{\mathrm{max}}} \mathbb{ZNR}
% \]
% $E_{\mathrm{max}}$ is the maximum transferable energy in a single
% collision with
% an atomic electron.
% \[
% E_{\mathrm{max}} =\frac{2 m_{\mathrm{e}} \beta^2\gamma^2 }{1 +
% 2\gamma m_{\mathrm{e}}/m_{\mathrm{x}} + \left ( m_{\mathrm{e}}
% /m_{\mathrm{x}}\right)^2}\ ,
% \]
% where $\gamma = E/m_{\mathrm{x}}$, $E$ is energy and
% $m_{\mathrm{x}}$ the mass of the incident particle,
% $\beta^2 = 1 - 1/\gamma^2$ and $m_{\mathrm{e}}$ is the electron mass.
% $\xi$ comes from the Rutherford scattering cross section
% and is defined as:
% \begin{eqnarray*} \xi  = \frac{2\pi z^2 e^4 N_{\mathrm{Av}} Z \rho
% \delta x}{m_{\mathrm{e}} \beta^2 c^2 A} =  153.4 \frac{z^2}{\beta^2}
% \frac{Z}{A}
%   \rho \delta x \quad\mathrm{keV},
% \end{eqnarray*}
% where
% 
% \begin{tabular}{ll}
% $z$          & charge of the incident particle \\
% $N_{\mathrm{Av}}$     & Avogadro's number \\
% $Z$          & atomic number of the material \\
% $A$          & atomic weight of the material \\
% $\rho$       & density \\
% $ \delta x$  & thickness of the material \\
% \end{tabular}
% 
% $\kappa$ measures the contribution of the collisions with energy
% transfer close to $E_{\mathrm{max}}$.  For a given absorber, $\kappa$
% tends
% towards large values if $\delta x$ is large and/or if $\beta$ is
% small.  Likewise, $\kappa$ tends towards zero if $\delta x $ is small
% and/or if $\beta$ approaches $1$.
% 
% The value of $\kappa$ distinguishes two regimes which occur in the
% description of ionisation fluctuations:
% 
% \begin{enumerate}
% \item A large number of collisions involving the loss of all or most
%   of the incident particle energy during the traversal of an absorber.
% 
%   As the total energy transfer is composed of a multitude of small
%   energy losses, we can apply the central limit theorem and describe
%   the fluctuations by a Gaussian distribution.  This case is
%   applicable to non-relativistic particles and is described by the
%   inequality $\kappa > 10 $ (\ie, when the mean energy loss in the
%   absorber is greater than the maximum energy transfer in a single
%   collision).
% 
% \item Particles traversing thin counters and incident electrons under
%   any conditions.
% 
%   The relevant inequalities and distributions are $ 0.01 < \kappa < 10
%   $,
%   Vavilov distribution, and $\kappa < 0.01 $, Landau distribution.
% \end{enumerate}
% 
% 
% \section{Various Mathematical Examples}
% If $n > 2$, the identity
% \[
%   t[u_1,\dots,u_n] = t\bigl[t[u_1,\dots,u_{n_1}], t[u_2,\dots,u_n]
%   \bigr]
% \]
% defines $t[u_1,\dots,u_n]$ recursively, and it can be shown that the
% alternative definition
% \[
%   t[u_1,\dots,u_n] = t\bigl[t[u_1,u_2],\dots,t[u_{n-1},u_n]\bigr]
% \]
% gives the same result.  

%*****************************************
%*****************************************
%*****************************************
%*****************************************
%*****************************************
