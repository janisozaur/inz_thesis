%************************************************
\myChapter{Aktualny stan zagadnienia}\label{ch:current_state} % $\mathbb{ZNR}$
%************************************************

Od kilku lat można zaobserwować wprowadzanie na rynek masowy, głównie jako elementów konsol do gier, systemów śledzenia ruchu.

Zadaniem takiego systemu jest śledzenie użytkownika w pewnej ograniczonej przestrzeni, w szczególności jego ruchów, oraz przetwarzanie pozyskanych danych na potrzeby gry wideo. W efekcie gracz wykazuje większe zaangażowanie w grę, ponieważ staje się jej aktywną częścią.

Pierwsze próby śledzenia ruchów podjęła firma \textsmaller{abrams gentile entertainment} w roku 1989 wprowadzając rękawicę \textsmaller{Power Glove}, która miała na celu śledzenie ruchów ręki użytkownika i wykorzystanie tych danych w konsoli \index{Nintendo!NES}\textsmaller{Nintendo Entertainment System}.

Z powodu słabego przyjęcia się urządzenia (miała na to wpływ między innymi jakość urządzenia oraz ilość dostępnych gier wykorzystujących możliwości tego kontrolera) kolejne próby śledzenia ruchów podjęła firma \index{Sony}\textsmaller{Sony} wprowadzając kamerę \textsmaller{EyeToy} dopiero w roku 2003. Ów odstęp czasu był długi również z powodu wymagań technologicznych, jakim musiał sprostać sprzęt aby nie tylko generować, ale także pobierać i przetwarzać obraz w czasie rzeczywistym. Wspomniane urządzenie także nie cieszyło się dużą popularnością.

Pomimo chłodnego przyjęcia dotychczasowych rozwiązań, producenci najwyraźniej dostrzegali duże możliwości rozwoju w tej dziedzinie, gdyż w~roku 2006. wprowadzona została przez firmę \index{Nintendo}\textsmaller{Nintendo} konsola \index{Nintendo!Wii}\textsmaller{Wii}, której głównym atutem był nowy sposób sterowania \ppauza za pomocą bezprzewodowego kontrolera wyposażonego w \index{akcelerometr}akcelerometr. Urządzenie to jest w stanie odczytywać przyspieszenia, jakie na nie działają we wszystkich trzech osiach. Wprowadzony w roku 2009. dodatek \textsmaller{MotionPlus}, który zawiera trójosiowy \index{żyroskop}żyroskop, umożliwia dodatkowo śledzenie obrotów w trzech osiach, znacznie zwiększając dokładność systemu.

Na dzień dzisiejszy, wśród aktualnych ,,dużych''\graffito{Mowa tutaj o~systemach, które nie są projektowane jako przenośne, chociaż i w tym przypadku zaczyna się wykorzystywać jakieś formy śledzenia ruchów.} konsol siódmej generacji, każdy z domowych systemów posiada system śledzenia ruchu.

Przybliżę ich charakterystykę zgodnie z kolejnością wprowadzania na rynek.

\index{Nintendo!Wiimote}\paragraph{Wiimote}
Kontroler do konsoli \textsmaller{Nintendo Wii}, wyposażony we wspomniany akcelerometr w technologii \graffito{MEMS \ppauza Microelectromechanical systems.}MEMS. Dzięki wprowadzeniu takiego kontrolera jako integralnej części systemu, firma \textsmaller{Nintendo} dokonując \graffito{O małej ilości zmian stanowi relatywnie małe zwiększenie mocy obliczeniowej, niewielkie różnice w~architekturze konsoli oraz pełna kompatybilność wsteczna (wliczając w to gry i akcesoria).} niewielkich zmian w istniejącym produkcie \ppauza \index{Nintendo!GameCube}\textsmaller{Nintendo GameCube} \ppauza lecz wprowadzając niespotykany dotychczas i bardzo naturalny sposób interakcji osiągnęła rynkowy sukces \citep{WiiSales}.

Kontroler posiada ponadto kamerę zdolną śledzić do czterech punktów, zaś zastosowany do kamery filtr zapewnia wysoką skuteczność śledzenia promieniowania podczerwonego. Element \textsmaller{Sensor Bar} umieszczany nad lub pod telewizorem, wyposażony w zestaw podczerwonych diod na każdym z końców, pozwala kontrolerowi na poznanie odległości od telewizora poprzez wyznaczenie względnego odstępu pomiędzy krańcami elementu \textsmaller{Sensor Bar}.

Ten mechanizm sterowania zyskał szczególną popularność dzięki grom, które odwzorowywały sporty, w których wykorzystywane są ,,drążki'' \ppauza np. baseball, golf, tenis.

Zalety:
\begin{itemize}
  \item jako integralna część konsoli jest dostarczany wraz z nią i nie odbiera zasobów programiście,
  \item wykorzystuje standard \texttt{I$^2$C},
  \item wykorzystuje standard Bluetooth.
\end{itemize}

Wady:
\begin{itemize}
  \item dostarcza jedynie ograniczone dane,
  \item nie jest możliwe wykorzystanie wszystkich możliwości kontrolera jednocześnie,
  \item wbudowana kamera ma mały kąt widzenia,
  \item ograniczenie śledzenia tylko do rąk.
\end{itemize}

\index{Sony!PlayStation!Eye}\paragraph{PlayStation Eye}
Dodatek do konsoli \textsmaller{Sony PlayStation 3} pod postacią kamery. Kamera ta jest w stanie nadawać obraz wymiarów 320\texttimes{}240 pikseli z częstotliwością 120Hz lub 640\texttimes{}480 pikseli i 60Hz; posiada także cztery mikrofony.

Urządzenie pozwala na wykrywanie i śledzenie ruchów za pomocą metod wizualnych oraz dźwiękowych, a także tłumienie echa. W zależności od programu, wykorzystanie kamery jest różne \ppauza może wahać się od pobrania zdjęcia użytkownika, przez śledzenie głowy, aż do pełnego śledzenia ciała. Jako że jest to zwykła kamera, ograniczona ona jest do zaledwie dwóch wymiarów. Imitację trzeciego można zaimplementować badając rozmiar śledzonego obiektu, rozwiązanie takie będzie jednak bardzo mało podatne na zmiany ,,głębokości''.

Oprogramowanie konsoli pozwala także na prowadzenie wideorozmów oraz prostą edycję nagranego materiału \citep{PSEye}.

Zalety:
\begin{itemize}
  \item dostarcza relatywnie dokładny (jak na standardy konsol) obraz z dużą częstotliwością,
  \item posiada zestaw mikrofonów zwiększający funkcjonalność.
\end{itemize}

Wady:
\begin{itemize}
  \item jako dodatek do systemu, a nie jego integralna część, jego oprogramowanie odbiera część zasobów dostępnych dotychczas dla programisty,
  \item śledzi ruch tylko w dwóch wymiarach.
\end{itemize}

\index{Nintendo!MotionPlus}\paragraph{Wii MotionPlus}
Jedno z rozszerzeń kontrolera \textsmaller{Wiimote}, które zawiera trójosiowy żyroskop. Urządzenie dzięki dostarczaniu danych o obrocie niweluje ograniczenia akcelerometru uniemożliwiające dokładne stwierdzenie rotacji.

Dane pobierane są z rozszerzenia za pomocą interfejsu \texttt{I$^2$C}, który pozwala na adresowanie do 127 urządzeń za pomocą zaledwie dwóch przewodów (oraz wspólnej masy), dzięki czemu możliwe jest dołączanie innych rozszerzeń, a także możliwe jest pozyskanie tych danych za pomocą mikrokontrolera.

Zwiększona precyzja daynch pozwoliła m.in. na dokładniejsze odwzorowanie sportów takich jak: łucznictwo, rzut dyskiem, golf oraz walka mieczem.

Zalety:
\begin{itemize}
  \item znacznie zwiększa precyzję dostarczanych danych.
\end{itemize}

Wady:
\begin{itemize}
  \item jako element opcjonalny powoduje fragmentację rynku.
\end{itemize}

\index{Sony!PlayStation!Move}\paragraph{PlayStation Move}
Rozszerzenie do konsoli \textsmaller{Sony PlayStation 3}, które opiera swoje działanie o wizualne metody śledzenia, dostępne za pomocą opisanego już \textsmaller{PlayStation Eye}.

System \textsmaller{PlayStation Move} składa się z dwóch kontrolerów: \textsmaller{Motion Controller} oraz \textsmaller{Navigation Controller}.

Kontroler \textsmaller{Motion Controller} zawiera sferę, która podświetlana jest za pomocą diod na kolor, jaki nie występuje w otoczeniu. Pozwala to na dokładne śledzenie położenia kontrolera przez kamerę, natomiast czujniki przyspieszenia, obrotu oraz geomagnetyczności zawarte w urządzeniu zapewniają wystarczającą ilość danych, jaka jest potrzebna do odwzorowania skomplikowanych ruchów. Element ten posiada także część przycisków dostępnych w standardowym kontrolerze \textsmaller{Sixaxis}.

Element \textsmaller{Navigation Controller} zawiera tylko przyciski oraz gałkę analogową.

Sterowanie takie wykorzystane jest w grach typu \textsl{First Person Shooter}, natomiast dema prezentowane przez firmę \textsmaller{Sony} obejmują łucznictwo, walki mieczami, boks i inne.

Zalety:
\begin{itemize}
  \item system automatycznie dostraja się do otoczenia,
  \item mnogość czujników zapewnia możliwość dokładnego odwzorowania ruchu,
  \item wykorzystuje standard Bluetooth.
\end{itemize}

Wady:
\begin{itemize}
  \item czujniki znajdują się tylko w kontrolerze \textsmaller{Motion Controller}, przez co wiele gier wymagających śledzenia obu rąk wymuszać będzie zakup więcej niż jednego zestawu,
  \item ograniczenie śledzenia tylko do rąk,
  \item przyciski dostępne w każdym z elementów osobno nie stanowią pełnej gamy przycisków oferowanych przez konsolę, co może powodować dodatkowy narzut projketowania i pracy programistycznej do obsługi obu schematów.
\end{itemize}

\index{Microsoft!Kinect}\paragraph{Kinect}
Dodatek do konsoli \textsmaller{Microsoft Xbox 360}, który zawiera kamerę RGB oraz czujnik 3D, którego działanie oparte jest na emitowaniu światła podczerwonego i odbieraniu jego odbicia. W urządzeniu zawarta jest także siatka mikrofonów (ang. \textsl{multi-array microphone}) oraz ruchoma podstawa. Dzięki tym elementom możliwe jest odpowiednio: śledzenie głosu i anulowanie echa oraz ,,skanowanie'' przestrzenie w celu dopasowania do użytkownika.

Gry wykorzystujące ten sensor pozwalają tańczyć, wykonywać gimnastykę, sporty a także wiele innych.

Sensor wykorzystywany jest także w wielu projektach, które nie dotyczą konsoli. Przykładowy projekt, dzięki wykorzystaniu dwóch urządzeń \textsmaller{Kinect} rekonstruuje trójwymiarową scenę w postaci wirtualnej, nakłada na nią pobrane na żywo dane z kamer RGB oraz pozwala przesuwać kamerę w wirtualnej scenie w miejsce, w którym nie jest zlokalizowany żaden z sensorów, nie tracąc przy tym szczegółów sceny \citep{TwoKinects}.

Zalety:
\begin{itemize}
  \item użytkownik nie musi trzymać żadnego urządzenia,
  \item system jest w stanie sam znaleźć użytkownika dzięki zastosowaniu ruchomej podstawy,
  \item może śledzić całe ciało,
  \item posiada sterowniki \textsl{open source}.
\end{itemize}

Wady:
\begin{itemize}
  \item ponieważ użytkownik nie trzyma żadnego kontrolera, nie ma możliwości naciśnięcia przycisku,
  \item nie jest możliwe prawdziwe odwzorowanie chodu, gdyż użytkownik porusza się w ograniczonej przestrzeni.
\end{itemize}


\paragraph{Podsumowanie}
Jak pokazuje rynek, istnieje wiele metod śledzenia ruchów gracza. Każda z nich ma odmienne wady i zalety, różne jest także ich zastosowanie.

Należy zauważyć, że wprowadzenie systemu śledzenia ruchów jako dodatku do konsoli stawia większe wymagania, niż ma to w przypadku zintegrowania takiego systemu wraz z nową konsolą:
\begin{itemize}
  \item w\graffito{Zasoby dostępne w~konsolach są mocno ograniczone, nie jest możliwe ich rozszerzanie.} celu obsługi systemu konieczne jest odebranie programiście pewnej ilości zasobów konsoli,
  \item konieczna jest promocja urządzenia,
  \item powodowana jest fragmentacja rynku, objawiająca się faktem, że użytkownik może nie posiadać danego dodatku, co odbija się na oprogramowaniu \ppauza nie wszystkie produkcje będą wykorzystywać system,
  \item często\graffito{Sytuacja taka miała miejsce w przypadku gry MAG.} koszta ponoszą studia deweloperskie, które dodają obsługę systemu już po premierze gry.
\end{itemize}

Nie jest możliwe stwierdzenie, który z tych systemów jest ,,najlepszy'', gdyż pomimo faktu, że wszystkie są systemami śledzenia ruchów, to są one jednak od siebie bardzo odmienne, a to uniemożliwia ich jednoznaczne porównanie.

% Ei choro aeterno antiopam mea, labitur bonorum pri no. His no decore
% nemore graecis. In eos meis nominavi, liber soluta vim cu. Sea commune
% suavitate interpretaris eu, vix eu libris efficiantur.
% 
% \section{Some Formulas}
% Due to the statistical nature of ionisation energy loss, large
% fluctuations can occur in the amount of energy deposited by a particle
% traversing an absorber element\footnote{Examples taken from Walter
% Schmidt's great gallery: \\
% \url{http://home.vrweb.de/~was/mathfonts.html}}.  Continuous processes
% such as multiple
% scattering and energy loss play a relevant role in the longitudinal
% and lateral development of electromagnetic and hadronic
% showers, and in the case of sampling calorimeters the
% measured resolution can be significantly affected by such fluctuations
% in their active layers.  The description of ionisation fluctuations is
% characterised by the significance parameter $\kappa$, which is
% proportional to the ratio of mean energy loss to the maximum allowed
% energy transfer in a single collision with an atomic electron:
% \graffito{You might get unexpected results using math in chapter or
% section heads. Consider the \texttt{pdfspacing} option.}
% \[
% \kappa =\frac{\xi}{E_{\mathrm{max}}} \mathbb{ZNR}
% \]
% $E_{\mathrm{max}}$ is the maximum transferable energy in a single
% collision with
% an atomic electron.
% \[
% E_{\mathrm{max}} =\frac{2 m_{\mathrm{e}} \beta^2\gamma^2 }{1 +
% 2\gamma m_{\mathrm{e}}/m_{\mathrm{x}} + \left ( m_{\mathrm{e}}
% /m_{\mathrm{x}}\right)^2}\ ,
% \]
% where $\gamma = E/m_{\mathrm{x}}$, $E$ is energy and
% $m_{\mathrm{x}}$ the mass of the incident particle,
% $\beta^2 = 1 - 1/\gamma^2$ and $m_{\mathrm{e}}$ is the electron mass.
% $\xi$ comes from the Rutherford scattering cross section
% and is defined as:
% \begin{eqnarray*} \xi  = \frac{2\pi z^2 e^4 N_{\mathrm{Av}} Z \rho
% \delta x}{m_{\mathrm{e}} \beta^2 c^2 A} =  153.4 \frac{z^2}{\beta^2}
% \frac{Z}{A}
%   \rho \delta x \quad\mathrm{keV},
% \end{eqnarray*}
% where
% 
% \begin{tabular}{ll}
% $z$          & charge of the incident particle \\
% $N_{\mathrm{Av}}$     & Avogadro's number \\
% $Z$          & atomic number of the material \\
% $A$          & atomic weight of the material \\
% $\rho$       & density \\
% $ \delta x$  & thickness of the material \\
% \end{tabular}
% 
% $\kappa$ measures the contribution of the collisions with energy
% transfer close to $E_{\mathrm{max}}$.  For a given absorber, $\kappa$
% tends
% towards large values if $\delta x$ is large and/or if $\beta$ is
% small.  Likewise, $\kappa$ tends towards zero if $\delta x $ is small
% and/or if $\beta$ approaches $1$.
% 
% The value of $\kappa$ distinguishes two regimes which occur in the
% description of ionisation fluctuations:
% 
% \begin{enumerate}
% \item A large number of collisions involving the loss of all or most
%   of the incident particle energy during the traversal of an absorber.
% 
%   As the total energy transfer is composed of a multitude of small
%   energy losses, we can apply the central limit theorem and describe
%   the fluctuations by a Gaussian distribution.  This case is
%   applicable to non-relativistic particles and is described by the
%   inequality $\kappa > 10 $ (\ie, when the mean energy loss in the
%   absorber is greater than the maximum energy transfer in a single
%   collision).
% 
% \item Particles traversing thin counters and incident electrons under
%   any conditions.
% 
%   The relevant inequalities and distributions are $ 0.01 < \kappa < 10
%   $,
%   Vavilov distribution, and $\kappa < 0.01 $, Landau distribution.
% \end{enumerate}
% 
% 
% \section{Various Mathematical Examples}
% If $n > 2$, the identity
% \[
%   t[u_1,\dots,u_n] = t\bigl[t[u_1,\dots,u_{n_1}], t[u_2,\dots,u_n]
%   \bigr]
% \]
% defines $t[u_1,\dots,u_n]$ recursively, and it can be shown that the
% alternative definition
% \[
%   t[u_1,\dots,u_n] = t\bigl[t[u_1,u_2],\dots,t[u_{n-1},u_n]\bigr]
% \]
% gives the same result.  

%*****************************************
%*****************************************
%*****************************************
%*****************************************
%*****************************************
