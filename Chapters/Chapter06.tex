\myChapter{Testy}\label{ch:tests}
%************************************************

\section{Cele}

Głównym zadaniem przeprowadzonych testów było zbadanie zachowania stworzonego systemu i oprogramowania pod kątem wykorzystania w grach wideo\graffito{Wynika to ze studiowania na specjalności Technologii Gier i Symulacji Komputerowych}. Nie jest jednak trudno sobie wyobrazić inne dziedziny, w których można wykorzystać systemu, jak np.:
\begin{itemize}
 \item medycyna \ppauza manipulacja trójwymiarowymi obrazami skanów pacjentów,
 \item cyfrowe modelarstwo \ppauza szybkie prototypowanie modeli,
 \item tworzenie filmów \ppauza przechwytywanie ruchu.
\end{itemize}

Głównym atutem systemu w tych polach jest naturalny sposób manipulacji i~interakcji z komputerem ze względu na pobieranie dodatkowo \ppauza w odróżnieniu od standardowego zostawu klawiatura + myszka \ppauza trzeciego wymiaru.

\section{Metody}
Aby określić przydatność systemu, należy zbadać kilka czynników. Ich wagi będą różnić się, w zależności od planowanego wykorzystania, jednak można zauważyć, że należy dążyć do optymalizacji cech takich jak:
\begin{itemize}
 \item dokładność,
 \item stabilność,
 \item niezawodność,
 \item prostota użytkowania,
 \item kompatybliność z oprogramowaniem,
 \item szybkość.
\end{itemize}


\section{Wyniki}
% dokladnosc
\paragraph{Dokładność}
Dokładność systemu została już określona w sekcji \ref{section:precision}, pozwolę więc sobie jedynie przytoczyć obliczone parametry.

Istnieją dwa rodzaje ograniczeń: spowodowane stworzonym oprogramowaniem oraz spowodowane wykorzystanym sprzętem. Ominięcie tych pierwszych jest stosunkowo łatwe\graffito{Trzeba liczyć się z faktem, że można natknąć się na ograniczenia mikrokontrolera} i~wymaga tylko zmiany pewnych parametrów oraz ponowne zaprogramowanie układu. W przypadku drugiego typu ograniczeń należy spodziewać się konieczności modyfikacji układu, jeśli chciałoby się zwiększać dokładność.

Mikrokontroler jest w stanie mierzyć pozycję markera z dokładnością do 0.34mm\graffito{Równania odpowiednio \ref{eq:microcontroller_limit} i \ref{eq:sound_limit}}, natomiast wykorzystanie sygnałów ultradźwiękowych o częstotliowści 40kHz nakłada na system ograniczenie dokładności do 8.5mm, a zatem ograniczenie całego systemu wynosi 8.5mm.

Chociaż w wielu przypadkach taka dokładność jest zupełnie wystarczająca, pozostawia jednak trochę do życzenia.

% stabilnosc
\paragraph{Stabilność}
Pomimo przytoczonej powyżej, relatywnie małej, dokładności układ wykazuje pewną niestabilność, która objawia się drganiem odtworzonej pozycji markera. Widać to dobrze na rysunku \ref{fig:charter_shaky}, którego górna część pokazuje odtworzoną pozycję w trzech wymiarach. Wszystkie dane, a w szczególności wartość $Y$, wykazują kilkucentymetrowe drgania pomimo stabilnego umocowania zarówno układu jak i markera.

\begin{figure}
 \includegraphics[width=\textwidth]{gfx/charter_shaky.png}
 \caption[Wykres pozycji i odległości jednego z markerów]{Wykres pozycji i odległości od czujników jednego z~markerów w~stałej pozycji}
 \label{fig:charter_shaky}
\end{figure}

% niezawodność
\graffito{coś trzeba napisać o niezawodności}

% prostota użytkowania
\paragraph{Prostota użytkowania}
Starałem się zaprojektować system w taki sposób, aby korzystanie z niego było możliwie proste, jednak wymogi systemu powodują, że korzystanie z niego nie jest tak proste, jak ma to miejsce w przypadku systemów ,,konkurencyjnych''\graffito{Wiimote, Sixaxis, Kinect...}.

Aby móc w pełni wykorzystać oprogramowanie, a~w~szczególności program \textsmaller{Viewer}\graffito{ponazywać programy}, wymagane jest dokonanie kalibracji. Nie jest to czynność trudna, lecz wymóg wykonania jej każdorazowo po uruchomieniu może zniechęcać użytkowników.

Kalibracja ma na celu ustawienie wewnętrznych parametrów programu, które będą pozwalały na dokładniejsze odwzorowanie ruchów użytkownika. Na wykorzystywane parametry składają się długość ręki, pozycja ramienia i rotacja ramienia.

W zależności od trybu kalibracji wymagane jest pobranie dwóch lub trzech punktów kontrolnych; w tym celu należy:
\begin{enumerate}
 \item jeśli została już dokonana kalibracja od uruchomienia programu, należy ją zresetować za pomocą stosownego przycisku,
 \item ustawić rękę (marker) w ustalonej pozycji względem ciała,
 \item nacisnąć przycisk informujący program o ustawieniu markera,
 \item powtórzyć czynność dla pozostałych punktów kontrolnych,
 \item gdy zostaną pobrane wszystkie punkty kontrolne, należy nacisnąć przycisk wyzwalający kalibrację, co spowoduje ustawienie trybu skalibrowanego.
\end{enumerate}

Wykonanie tej procedury spowoduje przekształcanie układu współrzędnych urządzenia na układ współrzędnych związany z użytkownikiem.

% kompatybilność z oprogramowaniem
\paragraph{Kompatybilność z oprogramowaniem}
Program wykorzystuje do komunikacji z komputerem protokół \textsmaller{RS232}. Jest to standard komunikacji szeregowej niegdyś bardzo często dostępny w komputerach, jednak dziś odpowiednie złącza i interfejsy dostępne są za sprawą stosunkowo niedrogich adapterów wpinanych do portu \textsmaller{USB}.

Chociaż do uruchomienia komunikacji z wykorzystaniem modułu \textsmaller{USART}\graffito{\textsmaller{USART} \ppauza Universal Synchronous and Asynchronous serial Receiver and Transmitter} mikrokontrolera nie jest wymagane stosowanie zewnętrznego oscylatora, to jego wykorzystanie zmniejsza ilość błędów, jakie mogą zachodzić podczas transmisji.

Ze względu na brak możliwości zakupu stosownego oscylatora, zdecydowałem się wykorzystać wbudowany w mikrokontroler zadajnik częstotliowści. W takiej konfiguracji, w przypadku transmisji danych z prędkością 9600bps\graffito{bps \ppauza baud per second} błąd wynosi zaledwie 0.2\%.

Moje doświadczenie z mikrokontrolerami pokazuje, że wartość ta jest dostatecznie mała, aby można było założyć, że trasmisja jest dokładna.

Wykorzystany do komunikacji interfejs oraz zaproponowany protokół\graffito{opisać protokół} jest bardzo prosty, co pozwala na dodanie obsługi systemu do dowolnego programu pragnącego skorzystać z tej możliwości znikomym nakładem pracy.

Stanowiłoby to duży atut systemu w przypadku jego popularyzacji.

\paragraph{Szybkość}
Niestety, szybkość układu pozostawia wiele do życzenia.

Istnieje kilka czynników, które można podejrzewać o spowalnianie działania. Rozpatrzmy po kolei każdy z nich.

\textsl{Wydajność obliczeń}\graffito{Obliczenia dotyczą głównie przekształcania układu współrzędnych} \ppauza aby nie obciążać zbędną pracą mikrokontrolera, który na dodatek nie posiada wystarczającej mocy obliczeniowej, obliczenia dokonywane są dopiero na komputerze\graffito{Jeśli dana aplikacja faktycznie takich obliczeń wymaga}. Chociaż w celu pełnego przywrócenia układu współrzędnych użytkownika z dowolnego ustawienia odbiorników wymaganych jest kilka obrotów oraz translacji punktów, a także obliczanie wartości odwrotnych funkcji trygonometrycznych oraz normalizacje trójskładnikowych wektorów, to wykorzystanie procesora przez aplikację zarówno na laptopie wyposażonym w procesor \textsmaller{Intel Core 2 Duo T7700 2.40GHz} oraz komputerze stacjonarnym z procesorerm \textsmaller{Intel Core 2 Duo E6420 2.13GHz} oscyluje w granicach kilku procent.

Przyjmuję więc, że praca ta wykonywana jest na tyle szybko, żeby nie spowalniała działania całego systemu\graffito{dodać testy wydajności - ilość pełnych transformacji układu na sekundę}.

